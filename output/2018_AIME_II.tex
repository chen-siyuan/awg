\documentclass{article}%
\usepackage[T1]{fontenc}%
\usepackage[utf8]{inputenc}%
\usepackage{lmodern}%
\usepackage{textcomp}%
\usepackage{lastpage}%
\usepackage{amsmath}%
\usepackage{hyperref}%
%
\title{2018 AIME II Problems}%
\author{MAA}%
\date{\today}%
%
\begin{document}%
\normalsize%
\maketitle%
\section{Acknowledgement}%
\label{sec:Acknowledgement}%
All the following problems are copyrighted by the \href{https://www.maa.org/}{Mathematical Association of America}'s \href{https://www.maa.org/math-competitions}{American Mathematics Competitions}.

%
\section{Problems}%
\label{sec:Problems}%
\begin{enumerate}%
\item%
[\textbf{Problem 1}]Points $A$, $B$, and $C$ lie in that order along a straight path where the distance from $A$ to $C$ is $1800$ meters. Ina runs twice as fast as Eve, and Paul runs twice as fast as Ina. The three runners start running at the same time with Ina starting at $A$ and running toward $C$, Paul starting at $B$ and running toward $C$, and Eve starting at $C$ and running toward $A$. When Paul meets Eve, he turns around and runs toward $A$. Paul and Ina both arrive at $B$ at the same time. Find the number of meters from $A$ to $B$.
%
\item%
[\textbf{Problem 2}]Let $a_{0} = 2$, $a_{1} = 5$, and $a_{2} = 8$, and for $n > 2$ define $a_{n}$ recursively to be the remainder when $4$($a_{n-1}$ $+$ $a_{n-2}$ $+$ $a_{n-3}$) is divided by $11$. Find $a_{2018}$ $\cdot$ $a_{2020}$ $\cdot$ $a_{2022}$.
%
\item%
[\textbf{Problem 3}]Find the sum of all positive integers $b < 1000$ such that the base-$b$ integer $36_{b}$ is a perfect square and the base-$b$ integer $27_{b}$ is a perfect cube.
%
\item%
[\textbf{Problem 4}]In equiangular octagon $CAROLINE$, $CA = RO = LI = NE =$ $\sqrt{2}$ and $AR = OL = IN = EC = 1$. The self-intersecting octagon $CORNELIA$ encloses six non-overlapping triangular regions. Let $K$ be the area enclosed by $CORNELIA$, that is, the total area of the six triangular regions. Then $K =$ $\dfrac{a}{b}$, where $a$ and $b$ are relatively prime positive integers. Find $a + b$.
%
\item%
[\textbf{Problem 5}]Suppose that $x$, $y$, and $z$ are complex numbers such that $xy = -80 - 320i$, $yz = 60$, and $zx = -96 + 24i$, where $i$ $=$ $\sqrt{-1}$. Then there are real numbers $a$ and $b$ such that $x + y + z = a + bi$. Find $a^2 + b^2$.
%
\item%
[\textbf{Problem 6}]A real number $a$ is chosen randomly and uniformly from the interval $[-20, 18]$. The probability that the roots of the polynomial
%
\item%
[\textbf{Problem 7}]Triangle $ABC$ has side lengths $AB = 9$, $BC =$ $5\sqrt{3}$, and $AC = 12$. Points $A = P_{0}, P_{1}, P_{2}, ... , P_{2450} = B$ are on segment $\overline{AB}$ with $P_{k}$ between $P_{k-1}$ and $P_{k+1}$ for $k = 1, 2, ..., 2449$, and points $A = Q_{0}, Q_{1}, Q_{2}, ... , Q_{2450} = C$ are on segment $\overline{AC}$ with $Q_{k}$ between $Q_{k-1}$ and $Q_{k+1}$ for $k = 1, 2, ..., 2449$. Furthermore, each segment $\overline{P_{k}Q_{k}}$, $k = 1, 2, ..., 2449$, is parallel to $\overline{BC}$. The segments cut the triangle into $2450$ regions, consisting of $2449$ trapezoids and $1$ triangle. Each of the $2450$ regions has the same area. Find the number of segments $\overline{P_{k}Q_{k}}$, $k = 1, 2, ..., 2450$, that have rational length.
%
\item%
[\textbf{Problem 8}]A frog is positioned at the origin of the coordinate plane. From the point $(x, y)$, the frog can jump to any of the points $(x + 1, y)$, $(x + 2, y)$, $(x, y + 1)$, or $(x, y + 2)$. Find the number of distinct sequences of jumps in which the frog begins at $(0, 0)$ and ends at $(4, 4)$.
%
\item%
[\textbf{Problem 9}]Octagon $ABCDEFGH$ with side lengths $AB = CD = EF = GH = 10$ and $BC = DE = FG = HA = 11$ is formed by removing 6-8-10 triangles from the corners of a $23$ $\times$ $27$ rectangle with side $\overline{AH}$ on a short side of the rectangle, as shown. Let $J$ be the midpoint of $\overline{AH}$, and partition the octagon into 7 triangles by drawing segments $\overline{JB}$, $\overline{JC}$, $\overline{JD}$, $\overline{JE}$, $\overline{JF}$, and $\overline{JG}$. Find the area of the convex polygon whose vertices are the centroids of these 7 triangles.
%
\item%
[\textbf{Problem 10}]Find the number of functions $f(x)$ from $\{1, 2, 3, 4, 5\}$ to $\{1, 2, 3, 4, 5\}$ that satisfy $f(f(x)) = f(f(f(x)))$ for all $x$ in $\{1, 2, 3, 4, 5\}$.
%
\item%
[\textbf{Problem 11}]Find the number of permutations of $1, 2, 3, 4, 5, 6$ such that for each $k$ with $1$ $\leq$ $k$ $\leq$ $5$, at least one of the first $k$ terms of the permutation is greater than $k$.
%
\item%
[\textbf{Problem 12}]Let $ABCD$ be a convex quadrilateral with $AB = CD = 10$, $BC = 14$, and $AD = 2\sqrt{65}$. Assume that the diagonals of $ABCD$ intersect at point $P$, and that the sum of the areas of triangles $APB$ and $CPD$ equals the sum of the areas of triangles $BPC$ and $APD$. Find the area of quadrilateral $ABCD$.
%
\item%
[\textbf{Problem 13}]Misha rolls a standard, fair six-sided die until she rolls 1-2-3 in that order on three consecutive rolls. The probability that she will roll the die an odd number of times is $\dfrac{m}{n}$ where $m$ and $n$ are relatively prime positive integers. Find $m+n$.
%
\item%
[\textbf{Problem 14}]The incircle $\omega$ of triangle $ABC$ is tangent to $\overline{BC}$ at $X$. Let $Y \neq X$ be the other intersection of $\overline{AX}$ with $\omega$. Points $P$ and $Q$ lie on $\overline{AB}$ and $\overline{AC}$, respectively, so that $\overline{PQ}$ is tangent to $\omega$ at $Y$. Assume that $AP = 3$, $PB = 4$, $AC = 8$, and $AQ = \dfrac{m}{n}$, where $m$ and $n$ are relatively prime positive integers. Find $m+n$.
%
\item%
[\textbf{Problem 15}]Find the number of functions $f$ from $\{0, 1, 2, 3, 4, 5, 6\}$ to the integers such that $f(0) = 0$, $f(6) = 12$, and
%
\end{enumerate}

%
\end{document}