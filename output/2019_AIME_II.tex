\documentclass{article}%
\usepackage[T1]{fontenc}%
\usepackage[utf8]{inputenc}%
\usepackage{lmodern}%
\usepackage{textcomp}%
\usepackage{lastpage}%
\usepackage{amsmath}%
%
\title{2019 AIME II Problems}%
\author{MAA}%
\date{\today}%
%
\begin{document}%
\normalsize%
\maketitle%
\section{Problems}%
\label{sec:Problems}%
\begin{enumerate}%
\item%
Two different points, $C$ and $D$, lie on the same side of line $AB$ so that $\triangle ABC$ and $\triangle BAD$ are congruent with $AB = 9$, $BC=AD=10$, and $CA=DB=17$. The intersection of these two triangular regions has area $\frac mn$, where $m$ and $n$ are relatively prime positive integers. Find $m+n$.
%
\item%
Lily pads $1,2,3,\ldots$ lie in a row on a pond. A frog makes a sequence of jumps starting on pad $1$. From any pad $k$ the frog jumps to either pad $k+1$ or pad $k+2$ chosen randomly with probability $\frac{1}{2}$ and independently of other jumps. The probability that the frog visits pad $7$ is $\frac{p}{q}$, where $p$ and $q$ are relatively prime positive integers. Find $p+q$.
%
\item%
Find the number of $7$-tuples of positive integers $(a,b,c,d,e,f,g)$ that satisfy the following systems of equations:
\begin{align*} abc&=70,\\ cde&=71,\\ efg&=72. \end{align*}
%
\item%
A standard six-sided fair die is rolled four times. The probability that the product of all four numbers rolled is a perfect square is $\frac{m}{n}$, where $m$ and $n$ are relatively prime positive integers. Find $m+n$.
%
\item%
Four ambassadors and one advisor for each of them are to be seated at a round table with $12$ chairs numbered in order $1$ to $12$. Each ambassador must sit in an even-numbered chair. Each advisor must sit in a chair adjacent to his or her ambassador. There are $N$ ways for the $8$ people to be seated at the table under these conditions. Find the remainder when $N$ is divided by $1000$.
%
\item%
In a Martian civilization, all logarithms whose bases are not specified as assumed to be base $b$, for some fixed $b\ge2$. A Martian student writes down
\[3\log(\sqrt{x}\log x)=56\]
\[\log_{\log x}(x)=54\]
and finds that this system of equations has a single real number solution $x>1$. Find $b$.
%
\item%
Triangle $ABC$ has side lengths $AB=120,BC=220$, and $AC=180$. Lines $\ell_A,\ell_B$, and $\ell_C$ are drawn parallel to $\overline{BC},\overline{AC}$, and $\overline{AB}$, respectively, such that the intersections of $\ell_A,\ell_B$, and $\ell_C$ with the interior of $\triangle ABC$ are segments of lengths $55,45$, and $15$, respectively. Find the perimeter of the triangle whose sides lie on lines $\ell_A,\ell_B$, and $\ell_C$.
%
\item%
The polynomial $f(z)=az^{2018}+bz^{2017}+cz^{2016}$ has real coefficients not exceeding $2019$, and $f\left(\frac{1+\sqrt{3}i}{2}\right)=2015+2019\sqrt{3}i$. Find the remainder when $f(1)$ is divided by $1000$.
%
\item%
Call a positive integer $n$ $k$-pretty if $n$ has exactly $k$ positive divisors and $n$ is divisible by $k$. For example, $18$ is $6$-pretty. Let $S$ be the sum of positive integers less than $2019$ that are $20$-pretty. Find $\frac{S}{20}$.
%
\item%
There is a unique angle $\theta$ between $0^{\circ}$ and $90^{\circ}$ such that for nonnegative integers $n$, the value of $\tan{\left(2^{n}\theta\right)}$ is positive when $n$ is a multiple of $3$, and negative otherwise. The degree measure of $\theta$ is $\frac{p}{q}$, where $p$ and $q$ are relatively prime integers. Find $p+q$.
%
\item%
Triangle $ABC$ has side lengths $AB=7, BC=8,$ and $CA=9.$ Circle $\omega_1$ passes through $B$ and is tangent to line $AC$ at $A.$ Circle $\omega_2$ passes through $C$ and is tangent to line $AB$ at $A.$ Let $K$ be the intersection of circles $\omega_1$ and $\omega_2$ not equal to $A.$ Then $AK=\frac mn,$ where $m$ and $n$ are relatively prime positive integers. Find $m+n.$
%
\item%
For $n \ge 1$ call a finite sequence $(a_1, a_2 \ldots a_n)$ of positive integers progressive if $a_i < a_{i+1}$ and $a_i$ divides $a_{i+1}$ for all $1 \le i \le n-1$. Find the number of progressive sequences such that the sum of the terms in the sequence is equal to $360$.
%
\item%
Regular octagon $A_1A_2A_3A_4A_5A_6A_7A_8$ is inscribed in a circle of area $1.$ Point $P$ lies inside the circle so that the region bounded by $\overline{PA_1},\overline{PA_2},$ and the minor arc $\widehat{A_1A_2}$ of the circle has area $\frac{1}{7},$ while the region bounded by $\overline{PA_3},\overline{PA_4},$ and the minor arc $\widehat{A_3A_4}$ of the circle has area $\frac{1}{9}.$ There is a positive integer $n$ such that the area of the region bounded by $\overline{PA_6},\overline{PA_7},$ and the minor arc $\widehat{A_6A_7}$ of the circle is equal to $\frac{1}{8}-\frac{\sqrt2}{n}.$ Find $n.$
%
\item%
Find the sum of all positive integers $n$ such that, given an unlimited supply of stamps of denominations $5,n,$ and $n+1$ cents, $91$ cents is the greatest postage that cannot be formed.
%
\item%
In acute triangle $ABC$ points $P$ and $Q$ are the feet of the perpendiculars from $C$ to $\overline{AB}$ and from $B$ to $\overline{AC}$, respectively. Line $PQ$ intersects the circumcircle of $\triangle ABC$ in two distinct points, $X$ and $Y$. Suppose $XP=10$, $PQ=25$, and $QY=15$. The value of $AB\cdot AC$ can be written in the form $m\sqrt n$ where $m$ and $n$ are positive integers, and $n$ is not divisible by the square of any prime. Find $m+n$.
%
\end{enumerate}

%
\end{document}