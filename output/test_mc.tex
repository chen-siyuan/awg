\documentclass{article}%
\usepackage[T1]{fontenc}%
\usepackage[utf8]{inputenc}%
\usepackage{lmodern}%
\usepackage{textcomp}%
\usepackage{lastpage}%
\usepackage{amsmath}%
\usepackage{hyperref}%
%
\title{2020 AMC 12 A Problems}%
\author{MAA}%
\date{\today}%
%
\begin{document}%
\normalsize%
\maketitle%
\section{Acknowledgement}%
\label{sec:Acknowledgement}%
All the following problems are copyrighted by the \href{https://www.maa.org/}{Mathematical Association of America}'s \href{https://www.maa.org/math-competitions}{American Mathematics Competitions}.

%
\section{Problems}%
\label{sec:Problems}%
\begin{enumerate}%
\item%
[\textbf{Problem 1}]Carlos took $70\%$ of a whole pie. Maria took one third of the remainder. What portion of the whole pie was left?

$\textbf{(A)}\ 10\%\qquad\textbf{(B)}\ 15\%\qquad\textbf{(C)}\ 20\%\qquad\textbf{(D)}\ 30\%\qquad\textbf{(E)}\ 35\%$

%
\item%
[\textbf{Problem 2}]The acronym AMC is shown in the rectangular grid below with grid lines spaced $1$ unit apart. In units, what is the sum of the lengths of the line segments that form the acronym AMC$?$

[asy] import olympiad; unitsize(25); for (int i = 0; i < 3; ++i) { for (int j = 0; j < 9; ++j) { pair A = (j,i);  } } for (int i = 0; i < 3; ++i) { for (int j = 0; j < 9; ++j) { if (j != 8) { draw((j,i)--(j+1,i), dashed); } if (i != 2) { draw((j,i)--(j,i+1), dashed); } } } draw((0,0)--(2,2),linewidth(2)); draw((2,0)--(2,2),linewidth(2)); draw((1,1)--(2,1),linewidth(2)); draw((3,0)--(3,2),linewidth(2)); draw((5,0)--(5,2),linewidth(2)); draw((4,1)--(3,2),linewidth(2)); draw((4,1)--(5,2),linewidth(2)); draw((6,0)--(8,0),linewidth(2)); draw((6,2)--(8,2),linewidth(2)); draw((6,0)--(6,2),linewidth(2)); [/asy]

$\textbf{(A) } 17 \qquad \textbf{(B) } 15 + 2\sqrt{2} \qquad \textbf{(C) } 13 + 4\sqrt{2} \qquad \textbf{(D) } 11 + 6\sqrt{2} \qquad \textbf{(E) } 21$

%
\end{enumerate}

%
\end{document}